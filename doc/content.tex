\section{Prerequisites}

\subsection{Naoqi workspace preparation}
\noindent It is recommended to work with workspaces when developing software for Softbank robots. This is because
Softbank uses qibuild for building the software developed for their robots. Hence, the following steps are recommended
to work with humoto-pepper-controller within a qibuild workspace (unless another workspace is to be used):

\begin{enumerate}
\item \texttt{mkdir /path/to/workspace/}
\item \texttt{cd /path/to/workspace/; qibuild init}
\item \texttt{cp -R /path/to/sdk /path/to/workspace/sdk}
\item \texttt{qitoolchain create mytoolchain /path/to/workspace/sdk/toolchain.xml}
\item \texttt{qibuild add-config mytoolchain -t mytoolchain --default}
\item \texttt{cd /path/to/workspace/; git clone <controller-pepper>}
\end{enumerate}

\noindent For the steps above it is assumed that qibuild is installed onto the system. It can be installed using for example pip
python packages manager.

\subsection{Compilation}
\noindent For compilation of the code the Aldebaran (Softbank Robotics) cross-compilation toolchain is reqired. The
controller has been developed using the toolchain of version: ctc-linux32-atom-2.4.3.28. Hence, it is recommended to use
this toolchain version. It can be obtained from the Aldebaran's official repositories (after registration).\\

\noindent To compile:
\noindent If your cross-compilation toolchain is named "mytoolchain":

- Type '`make controller-pepper`' inside the root directory of the project.

\noindent If your toolchain has a different name:

- Type '`make controller-pepper TC=name-of-your-toolchain`'.\\

\noindent Inside cmake input script pepper\_controller.cmake located inside cmake directory, user can enable or disable
options for using hot-starting and logging as well as specify the desired solver to be used. The following compilation
options are available:\\

\noindent CONTROLLER\_MPC\_HOTSTARTING\_ENABLED - this boolean option activates hotstarting for the MPC resolution.\\
\noindent CONTROLLER\_LOGGING\_ENABLED - this boolean option activates logging.\\
\noindent CONTROLLER\_HUMOTO\_MPC\_SOLVER\_NAMESPACE - this variable specifies the name of the solver used for MPC resolution. 
When using it make sure humoto is compiled with the respective solver.\\
\noindent CONTROLLER\_HUMOTO\_IK\_SOLVER\_NAMESPACE - this variable specifies the name of the solver used for IK resolution. 
When using it make sure humoto is compiled with the respective solver.\\

\noindent Inside the same directory user can also specify options for humoto compilation inside humoto.cmake file.
For possible options refer to the documentation of humoto.

\subsection{Installation}
\noindent The project compiles down to a shared object library (.so) file which is to be copied over onto the robot (for example
using scp). The library can be placed for example in the directory lib/ in the home directory of the nao user on the
robot. Once copied, the file should be declared inside the file 'autoload.ini' as follows:
'/path/to/libpeppercontroller.so'. After changing the 'autoload.ini' file naoqi operating system of the robot should be
rebooted to reread the file. This can be done using the command 'nao restart'.

\subsection{Usage}
Controller works with motion configuration files written in yaml language. Examples of such files are in the
<controller-pepper>/config/ directory. All files from this directory should be copied to \$HOME/config-pepper/ 
directory on the robot as this is the default path where the controller will look for the configuration files 
at runtime. Path of the directory with configuration files can be changed, however this change needs to be reflected in
the PepperController class contructor.

\subsection{Using visualization tool}
\noindent If compiled with logging enabled, one can make use of the visualization tool based off OpenSceneGraph
available on github: https://github.com/bip-team/osg-robot-visualizer.git.
